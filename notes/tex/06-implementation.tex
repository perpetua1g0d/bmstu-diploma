\chapter{Технологический раздел}

\textbf{Основные средства реализации}
\begin{enumerate}
\item k3d --- утилита для поднятия k8s кластера локально, использует docker, % https://k3d.io/stable/ 
\item kubectl --- утилита для ручного просмотра логов и состояния k8s кластера,
\item docker --- инструмент для контейнеризации приложений. Используется в реализации для создания sidecar контейнеров,
\item ghcr.io --- используется для загрузки docker образов в k8s кластер,
\item Golang --- язык программирования, в основном использующийся для написания приложений в микросервисной архитектуре. % todo: rewrite
% todo: либы jwt/jose ?
\end{enumerate}

%\section{Структура реализованного проекта}
% todo: диаграмма классов
На рисунке~\ref{img:tree} приведена структура реализованного проекта.
\includeimage
    {tree}
    {f}
    {H}
    {0.7\textwidth}
    {Структура проекта}


\section{Развертывание k8s кластера}
В листинге~\ref{lst:deploy.sh} приведен скрипт развертывания k8s сервисов. 
\includelisting
	{deploy.sh}{Скрипт развертывания k8s кластера}

В листингах~\ref{lst:a-deployment.yaml} и ~\ref{lst:talos-deployment.yaml} приведен пример  конфигурации сервиса вместе с сайдкаром в одном поде, а также конфигурация развертывания сервиса idP. 

\includelisting
	{a-deployment.yaml}{Конфигурация развертывания PostgreSQL сервиса с сайдкаром}

\includelisting
	{talos-deployment.yaml}{Конфигурация развертывания сервиса idP}

\section{Реализация idP сервиса}

idP сервис был реализован на языке программирования Golang и получил k8s кластере имя \textit{talos}.

В листингах ~\ref{lst:talos-cert-jwks.go}--\ref{lst:talos-claims.go} приведен способ генерации сертификата, основанного на rsa ключе, а также структура клеймов токена.

\includelisting
	{talos-cert-jwks.go}{Генерация сертификата}

\includelisting
	{talos-claims.go}{Структура клеймов токена}

В листингах ~\ref{lst:talos-cert-handler.go}--\ref{lst:talos-token.go} приведена реализация OIDC обработчиков HTTP запросов к сервису idP.
Обработчики слушают HTTP запросы на получение сертификатов по следующим путям:
\begin{enumerate}
\item ~\textit{/realms/infra2infra/.well-known/openid-configuration} --- обработчик запросов на получение OIDC конфигурации,
\item ~\textit{/realms/infra2infra/protocol/openid-connect/token} --- обработчик запросов на выпуск и получение токена idP,
\item ~\textit{/realms/infra2infra/protocol/openid-connect/certs} --- обработчик запросов на получение сертификатов idP.
\end{enumerate}

\includelisting
	{talos-cert-handler.go}{Реализация обработчика запросов на сертификаты}
	
\includelisting
	{talos-openid-config.go}{Реализация обработчика запросов на OIDC конфигурацию}	

\includelisting
	{talos-token.go}{Реализация обработчика запросов на выпуск токена}

В листингах ~\ref{lst:talos-k8s-jwks.go}--\ref{lst:talos-k8s-verifier.go} приведена реализация получения публичного k8s ключа сертификата проверки подписи k8s токена для последующего обмена на новый токен, который будет выпущен уже сервисом idP.

\includelisting
	{talos-k8s-jwks.go}{Получения публичного k8s ключа сертификата}

\includelisting
	{talos-k8s-verifier.go}{Проверка k8s токена}


\section{Реализация клиента к idP}
В листингах ~\ref{lst:auth-tokensource.go}--\ref{lst:auth-verifier.go} приведена реализация получения публичного сертификата idP и фонового получения токена для проверки токена входящего запроса.

\includelisting
	{auth-tokensource.go}{Фоновое получение idP токена}

\includelisting
	{auth-verifier.go}{Проверка idP токена}

Пример использования клиента к idP для проверки токена из входящего HTTP запроса в инфраструктурном сервисе приведен на листинге~\ref{lst:postgres-query.go}

\includelisting
	{postgres-query.go}{Проверка токена входящего запроса}

\section{Тестирование программного обеспечения}

Для функционального тестирования были написаны unit-тесты.
Пример unit теста для проверки k8s токена приведен в листинге~\ref{lst:talos-verifier_test.go}.

\includelisting
	{talos-verifier_test.go}{Тест для проверки k8s токена}

%\includelisting
%	{}{}
%	
%~\ref{img:}
%
% ~\cite{} 
% source-title
%
% image w/o extension
%\includeimage
%    {}
%    {f}
%    {H}
%    {1\textwidth}
%    {}
% last section is description