{\centering \chapter*{ВВЕДЕНИЕ}}
\addcontentsline{toc}{chapter}{ВВЕДЕНИЕ}

Последнее время все чаще можно услышать об утечке чувствительных персональных данных пользователей или компаний, занимающихся в таких сферах как FinTech, Healthcare, Enterprise. Отчасти это происходит из-за того, что компания и ее сотрудники небрежно следят за соблюдением информационной безопасности.

Информационная безопасность играет критически важную роль в распределенных системах с микросервисной архитектурой, поскольку такие системы часто состоят из множества независимых, взаимодействующих компонентов. Микросервисы обрабатывают и хранят большое количество данных, включая личную информацию пользователей и конфиденциальные бизнес-данные. Компрометация доступов к внутренним информационным системам и данным представляет собой финансовые, репутационные и правовые риски для компаний

В таких системах есть необходимость в эффективных и безопасных механизмах аутентификации и авторизации действий, доступных одному микросервису по отношению к другому, чтобы ограничить действия злоумышленника, получившего внутренний доступ. Особенно это касается инфраструктурных микросервисов --- там, где доступ к данным совершается наиболее часто.

В данной работе будет реализована система авторизации инфраструктурных сервисов в системах с микросервисной архитектурой, чтобы исключить один из возможных этапов утечки чувствительных данных --- несогласованный доступ как внутреннего сотрудника, так и злоумышленника извне систем. При этом работа авторизация не должна оказывать существенного влияния на работу системы, так как может быть внедрена в высоконагруженные системы.

\textbf{Цель} выпускной квалификационной работы --- реализация программно-алгоритмического комплекса системы авторизации инфраструктурных сервисов.

\textbf{Задачи} выпускной квалификационной работы:
\begin{enumerate}
\item провести обзор существующих подходов аутентификации и авторизации в микросервисной архитектуре;
\item рассмотреть основные протоколы аутентификации, применимые в микросервисной архитектуре;
\item разработать и описать ключевые алгоритмы работы программно-алгоритмического комплекса авторизации инфраструктурных микросервисов;
\item разработать программное обеспечение, реализующее аутентификацию и авторизацию инфраструктурных микросервисов;
\item провести исследование влияния работы авторизации на выполнения запросов между инфраструктурными сервисами.
\end{enumerate}

%В компаниях, в которых информационные системы разработаны и поддерживаются на микросервисной архитектуре, важно 

