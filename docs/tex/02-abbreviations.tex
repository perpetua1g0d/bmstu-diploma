{\centering \chapter*{РЕФЕРАТ}}
\addcontentsline{toc}{chapter}{РЕФЕРАТ}

Расчетно-пояснительная записка 62 с., 18 рис., 15 источн., 1 табл. \mbox{1 прил.}
% 0 табл.,

\noindent \mbox{АУТЕНТИФИКАЦИЯ}, \mbox{АВТОРИЗАЦИЯ}, \mbox{МИКРОСЕРВИСЫ}, \mbox{ИНФРАСТРУКТУРНЫЕ СЕРВИСЫ}, \mbox{KUBERNETES}, \mbox{КЛАСТЕР},  \mbox{SIDECAR}, \mbox{OIDC}, \mbox{OAUTH2.0}, \mbox{REST API},

Цель работы: реализация программно-алгоритмического комплекса для авторизации запросов в инфраструктурные сервисы.

{\centering \maketableofcontents}

{\centering \chapter*{ПЕРЕЧЕНЬ СОКРАЩЕНИЙ И ОБОЗНАЧЕНИЙ}}

В настоящей расчетно-пояснительной записке к выпускной квалификационной работе применяют следующие сокращения и обозначения:

% todo: fill + fix definitions
\begin{table}[H]
\begin{tabular}{p{5cm}p{10.5cm}}
k8s & Kubernetes --- программное обеспечение для оркестрирования контейнеризированных приложений --- автоматизации их развертывания, масштабирования и координации в условиях кластера
\tabularnewline
Sidecar & Паттерн сайдкар контейнера, при котором в одной сущности существует прокси-контейнер, расширяя возможности основного контейнера
\tabularnewline
IdP & Identity Provider --- ключевая точка авторизации, запросы аутентификации проходят через нее, и в ней же выписываются OIDC токены
\tabularnewline
Инфраструктурный сервис & сервис, представляющий из себя базовую инфраструктурную единицу, например база данных или брокер сообщений.
\tabularnewline
\end{tabular}
\end{table}