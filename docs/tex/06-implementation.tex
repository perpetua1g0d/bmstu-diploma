\chapter{Технологический раздел}

Основные средства реализации:
\begin{enumerate}
\item k3d --- утилита для поднятия k8s кластера локально, использует docker, % https://k3d.io/stable/ 
\item kubectl --- утилита для ручного просмотра логов и состояния k8s кластера,
\item docker --- инструмент для контейнеризации приложений. Используется в реализации для создания sidecar контейнеров,
\item ghcr.io --- используется для загрузки docker образов в k8s кластер,
\item Golang --- язык программирования, в основном использующийся для написания приложений в микросервисной архитектуре, % todo: rewrite
% todo: либы jwt/jose ?
\item Prometheus --- инструмент для сбора данных (метрик) сервисов,
\item Grafana --- инструмент для визуализации запросов к метрикам, а также для составления панелей визуализации с такими графиков.
\end{enumerate}

\section{Реализация IdP сервиса}

IdP сервису необходимо реализовать REST API, чтобы в него могли ходить за выпуском токена.
В соответствии с OIDC стандарт был реализован следующий контракт:

\begin{enumerate}
\item /realms/service2infra/.well-known/openid-configuration --- обработчик запросов на получение OIDC конфигурации. Конфигурация должна содержать адреса и пути обработчиков запросов на выпуск токена и получения сертификатов, адрес issuer --- где был выпущен токен, чтобы потом это проверить;
\item /realms/service2infra/protocol/openid-connect/token --- обработчик запросов на выпуск и получение токена idP. Параметрами для запроса могут быть \textit{"grant\_type"} --- способ проверки личности, в данном случае token-exchange, \textit{"subject\_token"} --- сам токен для обмена, \textit{"subject\_token\_type"} --- тип токена для обмена, \textit{"scope"} --- в какую сущность выписывается токен;
\item /realms/service2infra/protocol/openid-connect/certs --- обработчик запросов на получение сертификатов idP, возвращает публичный ключ сертификатов вместе с key-id.
\end{enumerate}

\pagebreak
В листинге~\ref{lst:talos-k8s-verifier.go} приведена реализация валидации k8s токена.
\includelisting
	{talos-k8s-verifier.go}{Валидация k8s токена}

В листинге~\ref{lst:talos-claims.go} приведены клеймы, которые будут дальше зашифрованы в токен.
\includelisting
	{talos-claims.go}{Клеймы выпускаемых токенов}
	
Сервис должен хранить свое состояние. В листинге~\ref{lst:talos-tables.sql} представлена таблица, в которой хранятся права.
\includelisting
	{talos-tables.sql}{Таблица для хранения прав}
	
\section{Реализация клиентской библиотеки}
Чтобы постоянно не выпускать токен на каждый запрос, было принято решение делать это фоново. В листинге~\ref{lst:client-tokensource.go} показана функция, запрашивающая токен в фоне.
\pagebreak
\includelisting
	{client-tokensource.go}{Фоновое обновление токенов}
	
В листинге~\ref{lst:client-verify.go} приведена реализация проверка подлинности токена.
\pagebreak
\includelisting
	{client-verify.go}{Проверка подлинности токен}

\section{Диаграмма компонентов разработанного ПО}
На рисунке~\ref{img:diag-comps} приведена диаграмма компонентов разработанного ПО. Как можно видеть, реализация подписи и проверки токенов запросов между сервисами вынесена в отдельный компонент так, чтобы сервисы не были напрямую зависимы от сервиса IdP.
\includeimage
    {diag-comps}
    {f}
    {H}
    {0.8\textwidth}
    {Диаграмма компонентов разработанного ПО}

\section{Тестирование программного обеспечения}

Для функционального тестирования были написаны unit-тесты.
Пример unit теста, реализованного с использованием Arrange-Act-Assert паттерна, для проверки k8s токена приведен в листинге~\ref{lst:talos-verifier_test.go}.

\pagebreak
\includelisting
	{talos-verifier_test.go}{Тест для проверки k8s токена}

Всего было покрыто 77\% кода функциональности ПО. На рисунке~\ref{img:coverage} представлено подробное покрытие.
\includeimage
    {coverage}
    {f}
    {H}
    {0.4\textwidth}
    {Покрытие тестами функциональности}
    
\section{Интерфейс ПО}
Для различного рода управления ПО был реализован интерфейс в виде административной панели. Пример интерфейса представлен на рисунке~\ref{img:interface}.
\includeimage
    {interface}
    {f}
    {H}
    {0.8\textwidth}
    {Пример интерфейса разработанного ПО}
    
Для наблюдения за состоянием системы была внедрен мониторинг ключевых показателей сервисов. Пример панели визуализации в Grafana с графиками, построенных по собранным данным, представлен на рисунке~\ref{img:main_dashboard}.
\includeimage
    {main_dashboard}
    {f}
    {H}
    {1\textwidth}
    {Пример панели визуализации по собранным данным}

\section*{Выводы по разделу}
В данном разделе были описаны средства реализации программного-алгоритмического комплекса, приведена примеры реализации IdP сервиса и клиентской библиотеки, результаты тестирования, а также интерфейс ПО и системы мониторинга.

%\includelisting
%	{}{}
%	
%~\ref{img:}
%
% ~\cite{} 
% source-title
%
% image w/o extension
%\includeimage
%    {}
%    {f}
%    {H}
%    {1\textwidth}
%    {}
% last section is description