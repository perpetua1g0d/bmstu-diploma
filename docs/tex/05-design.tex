\chapter{Конструкторский раздел}

\section{Метод авторизации HTTP запроса к инфраструктурному сервису}
На рисунке~\ref{img:a1} представлен формализованный в виде IDEF0 диаграммы метод авторизации HTTP запроса к инфраструктурному сервису
\includeimage
    {a1}
    {f}
    {H}
    {1\textwidth}
    {Формализованный метод авторизации HTTP запроса к инфраструктурному сервису}


\section{Алгоритм аутентификации инфраструктурного сервиса}
Для аутентификации сервиса-клиента, выполняющего подписанный запрос из сайдкар контейнера, необходим k8s токен пода, который хранится по пути с секретами. В обмен на этот k8s токен, IdP сервис выпускает токен аутентификации, который будет использоваться для подписи запроса от имени сервиса-клиента. На рисунке~\ref{img:alg-iss-client-token} представлен алгоритмы аутентификации сервиса-клиента.

\includeimage
    {alg-iss-client-token}
    {f}
    {H}
    {0.8\textwidth}
    {Алгоритм аутентификации инфраструктурного сервиса}

%\pagebreak

\section{Алгоритм верификации k8s токена на стороне IdP}
Для того, чтобы подтвердить подлинность k8s токена и личность, от имени которого он был выписан, IdP сервису нужен получить JWKs сертификаты от Kubernetes API, с помощью них расшифровать JWT токен и затем его проверить. Алгоритм верификации k8s токена представлен на рисунке~\ref{img:alg-k8s-verify}.

\includeimage
    {alg-k8s-verify}
    {f}
    {H}
    {0.6\textwidth}
    {Алгоритм верификации k8s токена}
    
%\pagebreak

\section{Алгоритм авторизации запроса на принимающей стороне}
На принимающей стороне запроса необходимо сначала достать из заголовка сам токен, и проверить его подпись с помощью сертификатов IdP. Затем необходимо проанализировать сам запрос на предмет того, какие роли необходимы для его авторизации, и если эти роли содержатся в токене, только тогда можно пропустить запрос. Алгоритм авторизации представлен на рисунке~\ref{img:alg-roles-verify}.

\includeimage
    {alg-roles-verify}
    {f}
    {H}
    {0.8\textwidth}
    {Алгоритм авторизации запроса на принимающей стороне}

%\pagebreak

%\section{Обработчики HTTP запросов к IdP}
%
%IdP сервису необходимо реализовать REST API, чтобы в него могли ходить за выпуском токена.
%OIDC стандарт предлагает следующий контракт:
%
%\begin{enumerate}
%\item /realms/service2infra/.well-known/openid-configuration --- обработчик запросов на получение OIDC конфигурации. Конфигурация должна содержать адреса и пути обработчиков запросов на выпуск токена и получения сертификатов, адрес issuer --- где был выпущен токен, чтобы потом это проверить;
%\item /realms/service2infra/protocol/openid-connect/token --- обработчик запросов на выпуск и получение токена idP. Параметрами для запроса могут быть \textit{"grant\_type"} --- способ проверки личности, в данном случае token-exchange, \textit{"subject\_token"} --- сам токен для обмена, \textit{"subject\_token\_type"} --- тип токена для обмена, \textit{"scope"} --- в какую сущность выписывается токен;
%\item /realms/service2infra/protocol/openid-connect/certs --- обработчик запросов на получение сертификатов idP, возвращает публичный ключ сертификатов вместе с key-id.
%\end{enumerate}

\section*{Вывод}
В данном разделе были спроектированы основные алгоритмы, необходимые для работы аутентификации.